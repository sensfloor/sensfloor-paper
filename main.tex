%% This is file `thws-mai-pm-template.tex'
%% Created: Magda Gregorova, 30/12/2023
%% Updated: Magda Gregorova, 06/01/2026
%%
%% This is a template for the final papers of the Project Module in the MAI program of the THWS.
%% The template is a minor modification of the SCORES conference template (https://www.scores.si/).
%% I hereby thank the original authors Klemen Berkovi\v{c}, Iztok Fister Jr., Iztok Fister and Luka F\"{u}rst
%% Any errors or problems in this template are my own doing. 
%% The template uses the acmart package https://ctan.org/pkg/acmart?lang=en

%% The first command in your LaTeX source must be the \documentclass command.
%% We use the review anonymous format for submissions - do not change this!
\documentclass[sigconf, review, nonacm, anonymous]{acmart}
% \documentclass[sigconf, nonacm]{acmart}

%% \BibTeX command to typeset BibTeX logo in the docs
\AtBeginDocument{\providecommand\BibTeX{{Bib\TeX}}}

%% Hack to allow for comments about author contribution - not used
% \newcommand{\projectcontrib}[1]{\affiliation{\country{#1}}}

%% If you need any other LaTeX packages, add them here
\usepackage{verbatim}
\usepackage[linesnumbered, boxed, resetcount]{algorithm2e}

%% For managing citations, it is recommended to use bibliography files in BibTeX format.
%%
%% You can then either use BibTeX with the ACM-Reference-Format style, or BibLaTeX with the acmnumeric or acmauthoryear sytles, that include support for advanced citation of software artefact from thebiblatex-software package, also separately available on CTAN.

%% print page numbers
\settopmatter{printfolios=true} 

%% end of the preamble, start of the body of the document source.
\begin{document}

%% The "title" command has an optional parameter allowing the author to define a "short title" to be used in page headers.
\title[\LaTeX\ Template for MAI Projects]{\LaTeX\ Template for Papers of Project Module in THWS.MAI Program}

%% The "author" command is used to define the authors.
%% You shall sabmit the anonymised version of your paper for review, so no need to include author names here.
% \author{First Awsomeauthor}
% \affiliation{\institution{THWS, MAI}\city{}\country{}}

% \author{Second Greatauthor}
% \affiliation{\institution{THWS, MAI}\city{}\country{}}

% \author{Third Amazingauthor}
% \affiliation{\institution{THWS, MAI}\city{}\country{}}

% %% By default, the full list of authors will be used in the page headers. 
% %% If this is too long use "shortauthors" to define a more concise list with surnames only.
% \renewcommand{\shortauthors}{Awsomeauthor, Greatauthor, Amazingauthor}

%% The abstract is a short summary of the work to be presented in the article.
\begin{abstract}
While camera-based systems are the dominant approach for human pose estimation, they face challenges in terms of privacy concerns and occlusion problems. These issues are of particular relevance in domains such as elderly care, where pose estimates can be used to monitor residents health or analyze incidents retrospectively. To assess an alternative to camera-based pose estimation, this paper aims to predict 3D walking poses using SensFloor: a capacitance-based floor which registers movement activity. We analyze the potential to utilize the floor's low-resolution signals to estimate poses and to what extent certain joint positions can be predicted accurately. For this purpose, we collected synchronized SensFloor signals and video data, from which we extracted 3D human poses using MediaPipe to serve as ground-truth targets for training. These signals and their corresponding targets were then used for supervised training of an LSTM neural network. To estimate the person's position on the floor, a Kalman filter was applied to smooth the noisy SensFloor measurements. Our results demonstrate that it is possible to predict human walking poses using the proposed methods, establishing a proof-of-concept for an alternative way of activity monitoring.
\end{abstract}

%% Keywords: pick words that accurately describe the work being presented. Separate the keywords with commas - not used.
% \keywords{\LaTeX, template, project, MAI}
%% This command processes the author and affiliation and title information and builds the first part of the formatted document.
\maketitle

%% Add here the link to your code repository
\vspace{-0.5em}
\textbf{Code:} \url{http://github.com/gitname/wonderful-repo}

\section{Introduction}
One part of the Project Module evaluation is a paper describing the results of the student's project.
The paper shall be written in \LaTeX\ to ensure high typesetting quality, correct formatting of mathematical symbols and formulas, and cross-platform compatibility.
This document provides the template for the papers that students shall use when writing the papers.

This \LaTeX\ template fixes the basic formatting requirements for the paper such as fonts, sizes, margins, spacing and the double-column format so that you do not have to worry about those.
As an author you only need to follow a few basic rules explained here below and ensure that the paper does not exceed \textbf{four pages}. 
Acknowledgments and references do not count into this limit and can be on the fifth page.
The \textbf{four pages} is a hard constraint, make sure you do not go over!
While camera-based systems are the dominant approach for human pose estimation, they face challenges in terms of privacy concerns and occlusion problems. These issues are of particular relevance in domains such as elderly care, where pose estimates can be used to monitor residents health or analyze incidents retrospectively. To assess an alternative to camera-based pose estimation, this paper aims to predict 3D walking poses using SensFloor: a capacitance-based floor which registers movement activity. We analyze the potential to utilize the floor's low-resolution signals to estimate poses and to what extent certain joint positions can be predicted accurately. For this purpose, we collected synchronized SensFloor signals and video data, from which we extracted 3D human poses using MediaPipe to serve as ground-truth targets for training. These signals and their corresponding targets were then used for supervised training of an LSTM neural network. To estimate the person's position on the floor, a Kalman filter was applied to smooth the noisy SensFloor measurements. Our results demonstrate that it is possible to predict human walking poses using the proposed methods, establishing a proof-of-concept for an alternative way of activity monitoring.
\section{Title Information}
The title of your work should use capital letters appropriately - \url{https://capitalizemytitle.com/} has useful rules for capitalization.
Use the {\verb|title|} command to define the title of your work.
Do not insert line breaks in your title.

If your title is lengthy, you must define a short version to be used in the page headers, to prevent overlapping text.
The \verb|title| command has a ``short title'' parameter:
\begin{verbatim}
  \title[short title]{full title}
\end{verbatim}

\section{Authors and Acknowledgements}
We use anonymised submissions for review, so do not include author names in your initial submission.
Author names and supervisors shall be added directly in the submission system and will not be visible to the reviewers.
In the submission system you shall also provide a brief description of the contribution of each author to the project and paper.
% All student project members shall be indicated as authors. 
% Names should not be abbreviated, use full names wherever possible.
% The following command should be placed just after the last \verb|\author{}| definition and updated with the actual author names:
% \begin{verbatim}
%   \renewcommand{\shortauthors}{...}
% \end{verbatim}
% Omitting this command will force the use of a concatenated list of all
% of the authors' full names, which may result in overlapping text in the
% page headers.
% Project supervisors (professor, PhD students, external advisors) shall not be indicated as authors. 
% Instead, these shall be acknowledged in the section \verb|Acknowledgments| at the end of the paper.
% Contribution of each author shall be indicated in the section \verb|Project Contribution| at the end of the paper.
% Describe here both the overall contribution of each author to the project as well as which sections of the paper were written by which author.

\section{Generative AI and LLM}
Use of generative AI tools such as large language models (LLMs) is becoming increasingly common in scientific writing and research.
Authors are welcome to use any tools that can improve the quality of their work, provided that they adhere to two basic principles: authors clearly document their methodology including the use of such tools to promote scientific rigour and transparency, and authors take full responsibility for the content and the quality of their work.
AI and LLMs are not eligible for authorship.
Instead any use of such tools should be described in the appropriate sections of the paper, e.g. in the Experiments section if the tool was used to generate data, or for LLM-based evaluation.
Use for spellchecking, grammar correction or basic programming assistance does not need to be mentioned. 

\section{The \emph{Body} of The Paper}
Typically, the body of a paper is organized into a hierarchical structure,
with numbered or unnumbered headings for sections, subsections,
sub-subsections, and possibly even smaller sections.  The command
\verb|\section{}| that precedes this paragraph is part of such
a hierarchy.  By using the appropriate sectioning commands, you make \LaTeX{}
handle the numbering and placement of the headings for you.  If you want a
sub-subsection or smaller part to be unnumbered in your output, simply append
an asterisk to the command name.  Examples of both numbered and unnumbered
headings will appear throughout this sample document.\footnote{This is a footnote. It adds nothing in
terms of content. It is meant to give you an idea of how footnotes look
and work. It is a wordy footnote, so you can see how a longish one plays
out.}

Since the entire article is contained in the \textbf{document} environment,
you can indicate the start of a new paragraph with a blank line in your input
file; that is why this sentence forms a separate paragraph.

\subsection{Type Changes and \emph{Special} Characters}

We have already seen several typeface changes in this sample.  You can
indicate \emph{italicized} words or phrases in your text with the command
\verb|\emph{}| or \verb|\textit{}|,
\textbf{boldface} text with the command \verb|\textbf{}|, and
\texttt{typewriter-style text} (e.g., for program code) with
\verb|\texttt{}|.  Remember, however, that you do not have to
indicate typestyle changes when such changes are part of the \emph{structural}
elements of your article; for instance, the heading of this subsection will be
in boldface, but that is handled by \LaTeX{} itself.

You can use whatever symbols, accented characters, or non-English characters
you need anywhere in your document.\footnote{Th is a second footnote. Let's make it
rather short to see how it looks.}  You can find a complete list of what is
available in the \emph{\LaTeX{} User's Guide}~\cite{Lamport:LaTeX}.

\subsection{Math Equations}

You may want to display math equations in three distinct styles: inline,
numbered display, or non-numbered display.  Each of these styles is discussed
in the next sections.

\subsubsection{Inline (In-text) Equations}

A formula that appears in the running text is called an inline or in-text
formula.  It is produced by the \textbf{math} environment, which can be
invoked with the usual
\verb|\begin{}|--\verb|\end{}| construct or with
the short form \textbf{\$$\ldots$\$}.  You can use any of the symbols and
structures, from $\alpha$ to $\omega$, available in
\LaTeX~\cite{Lamport:LaTeX}.  

The inline style is not completely equivalent to the display style. For
example, as we will see in the next section, the inline equation
\begin{math}\lim_{x\rightarrow \infty}\frac{1}{x}=0\end{math} 
looks slightly different when set in the display style.

\subsubsection{Display Equations}

A numbered display equation --- one set off by a vertical space from the text
and centered horizontally --- is produced by the \textbf{equation}
environment.  An unnumbered display equation is produced by the
\textbf{displaymath} environment.

Again, in either environment, you can use any of the symbols and constructs
available in \LaTeX; this section will just give a couple of examples of
display equations.  First, consider the equation shown as an inline equation
above:
%
\begin{equation}
\lim_{x\rightarrow \infty}\frac{1}{x}=0.
\end{equation}
%
Notice that it is formatted somewhat differently in the \textbf{displaymath}
environment.  Now, let us enter an unnumbered equation
%
\begin{displaymath}
    \sum_{i=1}^{\infty} \frac{1}{x^2} = \frac{\pi^2}{6}
\end{displaymath}
%
followed by another numbered equation:
%
\begin{equation}
    \int_{0}^{\pi/2} \cos x\,dx = \sin x\bigg\rvert_{0}^{\pi/2} = \sin
    \frac{\pi}{2} - \sin 0 = 1.
\end{equation}

Next you will see an example of an unnumbered equation that is not set in the
\textbf{displaymath} environment but in short form defined with
\textbf{\$\$$\ldots$\$\$}.  When $a \ne 0$, there are two solutions to $ax^2 +
bx + c = 0$, and they are
%
$$x_{1, 2} = \frac{-b \pm \sqrt{b^2-4ac}}{2a}.$$

Here is an example of referencing an equation. Equation~\eqref{equ:yannibel}
shows how to write cases in \LaTeX.
%
\begin{equation}
    \begin{aligned} 
        \mathrm{nr}(G_i,r) & = \label{equ:yannibel}
        \begin{cases}
            1  & \text{if $r$ is played by one member of $G_i$;}\\
            -2 & \text{if $r$ is not played in $G_i$;} \\
            -p & \text{if $r$ is played by $p$ members in $G_i$.}\\
        \end{cases}
    \end{aligned}
\end{equation}

\subsubsection{Long equations}

When an equation is too long for a single column, use the \textbf{aligned}
environment within the \textbf{equation} environment.  To align the equation
inside the \textbf{aligned} environment, use the symbol \textbf{\&} as seen in
Equation~\ref{equ:ho}.

\begin{equation}
    \begin{aligned}
        O_{\max}& = w_1 \sum_{a=1}^{m} \sum_{b=a+1}^{n} (-\lvert\text{CPT}_a 
        -\text{CPT}_b\rvert)\\ 
        &\quad + w_2 \sum_{j=1}^{m} (\text{DIF}_j) + w_3 \sum_{j=1}^{m} 
        (\text{INT}_j/\sum_{x=1}^{n} x_{ij})
    \end{aligned}
    \label{equ:ho}
\end{equation}

\subsection{Citations}

Citations to articles \cite{lecun2015deep, braams:babel, herlihy:methodology},
conference proceedings \cite{vrbancic2019transfer, clark:pct}, or books
\cite{salas:calculus, Lamport:LaTeX, fister2019computational} listed in the
Bibliography section of your article will probably occur throughout your text.
You should use \texttt{bibtex} to produce this bibliography automatically; you
simply have to insert one of several available citation commands with the key
of the item cited at the proper location in the \texttt{.tex} file
\cite{Lamport:LaTeX}.  The key is a short reference that you invent to
identify each work uniquely; in this sample document, the key is the first
author's surname and a word from the title.  This identifying key is included
with each item in the \texttt{.bib} file for your article.

The details of how to create the \texttt{.bib} file are beyond the scope of
this sample document.  More information can be found in the \emph{Author's
Guide}; for exhaustive details, see the \emph{\LaTeX{} User's
Guide}~\cite{Lamport:LaTeX}.

This article employs only the plainest form of citation, the one produced with
the \verb|cite{}| command.  This is, in fact, the only
citation style recommended by the ACM.

\subsection{Tables}

Since a table cannot be split across pages, we typically place it at the top
of the page, close to its initial reference.  To achieve a proper ``floating''
placement of tables, use the environment \textbf{table} to enclose the table's
contents and caption.  The contents of the table itself have to be put inside
the \textbf{tabular} environment, which ensures a suitable alignment of rows
and columns. Again, detailed instructions on \textbf{tabular} material can be
found in the \emph{\LaTeX{} User's Guide}.

Immediately following this sentence is the point at which
Table~\ref{tab:table1} is included in the input file; compare the placement of
the table here with the table in the PDF output of this document.

\begin{table}[h]
    \centering
    \caption{Frequency of Special Characters.}
    \label{tab:table1}
    \begin{tabular}{ccl}
        \toprule
        Non-English or Math&Frequency&Comments\\
        \midrule
        \O & 1 in 1,000& Swedish names\\
        $\pi$ & 1 in 5& In math\\ 
        \$ & 4 in 5 & In business\\ 
        $\Psi^2_1$ & 1 in 40,000& Unexplained\\
        \bottomrule
    \end{tabular}
\end{table}

To set a wider table (one that takes up the whole width of the page's live
area), use the environment \textbf{table*}.  As with a single-column table,
this wide table will ``float'' to a location deemed more desirable.
Immediately following this sentence is the point at which
Table~\ref{tab:table2} is included in the input file; again, it is instructive
to compare the placement of the table here with the table in the PDF output of
this document.

\begin{table*}
    \centering
    \caption{Some Typical Commands.}
    \label{tab:table2}
    \begin{tabular}{ccl} \hline
        \toprule
        Command&A Number&Comments\\
        \midrule
        \texttt{\textbackslash{}imagespath} & 200 & To provide the directory of included images \\
        \texttt{\textbackslash{}table} & 300 & For tables\\
        \texttt{\textbackslash{}table*} & 400& For wider tables\\
        \bottomrule
    \end{tabular}
\end{table*}

\subsection{Figures}

Like tables, figures cannot be split across pages; the best placement for them
is typically the top or the bottom of the page, close to their initial
reference\footnote{The fourth, and last, footnote.}.  To ensure a proper
``floating'' placement of figures, use the environment \textbf{figure} to
enclose the figure and its caption.

Figure~\ref{fig:circles} displays an image in the PDF format.
Figure~\ref{fig:star} shows a PNG image.

\begin{figure}
    \centering
    \includegraphics[scale=0.5]{img/star.png}
    \caption{A sample star graphic (PNG format).}
    \label{fig:star}
\end{figure}

As with tables, you may sometimes want a figure to span over two columns.  To
achieve this, while still ensuring a proper ``floating'' placement, use the
environment \textbf{figure*}.  An example can be seen in
Figure~\ref{fig:spin}.

\begin{figure*}
    \vspace{0.5cm} % we add small space, since table is being placed above the figure.
    \centering
    \includegraphics[scale=0.8]{img/spin.png}
    \caption{A sample spin graphic with a span.}
    \label{fig:spin}
\end{figure*}

\subsection{Lists}

In some cases, you might want to present your ideas using lists.  Lists are
created with the \textbf{itemize} environment.  In the next example, you can
see how lists are created and used:

\begin{itemize}
    \item First item,
    \item second item, and
    \item third item.
\end{itemize}

Sometimes, authors want to reference certain list items in the subsequent
text.  For that purpose, you can use the \textbf{enumerate} environment.
Following is an example of a numbered list:

\begin{enumerate}
    \item First point,
    \item\label{list:item} second point,
    \item $\ldots$
\end{enumerate}

Item \ref{list:item} tells us what to do once we check off the preceding item.

\subsection{Algorithms}

To display algorithms in your document, employ the \textbf{algorithm}
environment.  Algorithms can be referenced in the same way as tables and
figures (e.g., Algorithm~\ref{algo:sample}).

\begin{algorithm}
    \SetAlgoLined
    \KwData{this text}
    \KwResult{how to write an algorithm with \LaTeX}
    initialization\;
    \tcc{this is a comment to tell you that we will now really start the code}
    \While{not at end of this document}{\label{algo:sample:while}
        read the current section\;
        \eIf{understand}{
            go to the next section\;
            this section becomes the current section\;
        }{
            go back to the beginning of the current section\;
        }
    }
    \caption{How to write algorithms.}
    \label{algo:sample}
\end{algorithm}

You can reference any line of your algorithm: an example of the while loop can
be seen in line~\ref{algo:sample:while}.  For more details on the \textbf{algorithm}
environment, see the
\url{http://tug.ctan.org/macros/latex/contrib/algorithm2e/doc/algorithm2e.pdf}
document.
It is recommended to link the full code repository in the \verb|Code| line at the beginning of the paper.

\section{Conclusions}

This paragraph will end the body of this sample document.
There are two more things that follow: the bibliography and appendices.
The bibliography list is produced automatically using the \texttt{bibtex} command from the
\texttt{.bib} file and the citations in your text.  
You can use the appendices for additional material which does not belong or fit the main text.
To conclude, let us make
a disclaimer regarding the bibliography in this sample paper: with the
exception of the reference to the \LaTeX{} book, the citations in this paper
refer to works that have nothing to do with the present subject and are
used as examples only.

%% The next two lines define the bibliography style to be used, and
%% the bibliography file.
\bibliographystyle{ACM-Reference-Format}
\bibliography{references}

%% We use anonymised submissions for review, so do not include acknowledgements here.
% \section*{Acknowledgements}
% We thank Prof. Dr. Expert Inthefield for supervising our project and PhDStudent VeryHelpful for his valuable advice on many things we would otherwise never figure out.

%% If your work needs an appendix, this is the place to put it.
\appendix

\section{Appendices}\label{sec:Appendices}

Remember that your paper shall not exceed \textbf{four pages} (references can be on the fifth page).
If you need more space to present additional figures or tables, that are impossible to fit into the main text, you can use the appendix.
However, do not use the appendix to extend the length of your paper beyond the 4 page limit.
The appendix can be skipped when reading and evaluating your paper.  
Your paper shall be self-contained and completely understandable without the appendix.

Start the appendix with the ``\verb|\appendix|'' command
and note that in the appendix sections are lettered, not
numbered.
This document has two appendices, demonstrating the section
and subsection identification method.


\section{Second Section in Appendix}

Lorem ipsum dolor sit amet, consectetur adipiscing elit. Morbi
malesuada, quam in pulvinar varius, metus nunc fermentum urna, id
sollicitudin purus odio sit amet enim. Aliquam ullamcorper eu ipsum
vel mollis. Curabitur quis dictum nisl. Phasellus vel semper risus, et
lacinia dolor. Integer ultricies commodo sem nec semper.

\subsection{Some Subsection}

Etiam commodo feugiat nisl pulvinar pellentesque. Etiam auctor sodales
ligula, non varius nibh pulvinar semper. Suspendisse nec lectus non
ipsum convallis congue hendrerit vitae sapien. Donec at laoreet
eros. Vivamus non purus placerat, scelerisque diam eu, cursus
ante. Etiam aliquam tortor auctor efficitur mattis.

\end{document}
%% End of file `thws-mai-pm-template.tex'




