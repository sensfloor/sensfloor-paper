\section{Conclusion}
\label{sec:conclusion}
This study investigated the feasibility of estimating 3D walking poses from SensFloor signals.
Our results indicate that simple walking poses can be predicted, which could support privacy-preserving monitoring in applications such as elderly care.

The present work has several limitations. First, the model is currently tied to the patch size it was trained on and does not transfer to other SensFloor configurations with different patch sizes. Second, the evaluated walking poses are relatively homogeneous, and the dataset contains only three male participants. Training and evaluating on more diverse movements and a broader range of participants would be necessary to assess generalization. Third, the ground-truth poses obtained via MediaPipe are themselves subject to estimation error, which places an upper bound on the achievable accuracy.

Future work should therefore improve the transfer to other SensFloor patch sizes, incorporate more diverse data and use more accurate pose targets to better quantify the limits and potential of the approach.
