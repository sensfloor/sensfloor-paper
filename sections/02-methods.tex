% General overview of how to solve the problem (localization and pose prediction)

% Data collection (important: mention how our data looks like)
% - Who did we collect?
% - How does the recording setup look like?
% - Collect video and sensfloor at the same time -> Output: mp4 and csv
% - Creating labels for training using Mediapipe -> Output: Ground truth joint positions in coordinate system with hip as origin

% Describe training of the model
% - Model architecture: Supervised regression model
% - Data (roi, history)
% - Data split
% - Loss (link loss, MSE)
% - Training setup (config) and how we determined the hyperparameters -> Mention what joints we predict here (or in data collection)
% - Metrics (PCK, MJPE)
% - What is the output?
\subsection{Training Setup}
We used a CNN-LSTM architecture for training our model. It is based on the model from \todo{cite correctly}.
Instead of rotating the step, we use a CNN to account for the Translation invariance.
Figure STH shows the full architecture. 
Instead of giving the model all signals of the floor, we construct a Region of Interest (ROI)
which is a part of the floor containing the most signals in a defined sequence.
A sequence is a collection of frames. For each frame we have a state of the floor with certain signals.
Our model receives the signals in the ROI frame by frame and produces a vector with continuous values as logits.
These logits are given into our Loss function which can be mathematically described in the following
\todo{replace with correct equation}
We substract the predictions from the reference labels, square and multiply the weights of the landmarks.
The link loss is identical to the one in \todo{replace with correct cite}
We assume an upright position, this is why the upper body and the head will only rotate and the focus is more on the knee and feet
Mediapipe extracts 33 Joints, we only used 13, excluding the fingers, face expression, heels and foot index.
For each epoch we calculate the following metrics: Mean joint position error (MJPE), percentage correct keypoits (PCK) with a 5 and 10 threshold


We assume only one person is walking on the floor

% Describe localization
% - Sensor & clustering of signals
% - Kalman filter approach
% - Component values
% - What is the output?

% Describe how localization and pose prediction are combined -> Summary that connects training and localization outputs

% How to visualize floor and predicted pose
% - Setup for live predictions