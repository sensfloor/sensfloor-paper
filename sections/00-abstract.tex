While camera-based systems are the dominant approach for human pose estimation, they face several challenges in terms of privacy and occlusion effects.
These issues are of particular importance in domains such as elderly care, where it could be useful to monitor the residents to conclude about their health
conditions or backtrack incidents.
To assess an alternative for camera based pose estimation, this paper aims to predict walking poses using SensFloor: 
a capacitive-based floor which registers movement activity.
In this context we analyze whether it is possible to use the floors low resolution signal to fulfil this task and to what
extent certain joint positions can be predicted correctly. 

For this purpose, we collected synchronized data of SensFloor signals combined with a video capture. 
To extract human poses from the video as reference labels for training, we use MediaPipe's 3D pose estimation model.
The SensFloor signals together with the labels are used for supervised training of an LSTM neural network. 
To localize the poses on the floor, we use a Kalman filter which processes the same signals.

Our results demonstrate, that it is possible to predict human walking poses using the proposed methods,
establishing a proof-of-concept for an alternative way of activity monitoring.