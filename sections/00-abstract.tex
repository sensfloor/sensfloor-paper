While camera-based systems are the dominant approach for human pose estimation, they face challenges in terms of privacy concerns and occlusion problems. These issues are of particular relevance in domains such as elderly care, where pose estimates can be used to monitor residents health or analyze incidents retrospectively. To assess an alternative to camera-based pose estimation, this paper aims to predict 3D walking poses using SensFloor: a capacitance-based floor which registers movement activity. We analyze the potential to utilize the floor's low-resolution signals to estimate poses and to what extent certain joint positions can be predicted accurately. For this purpose, we collected synchronized SensFloor signals and video data, from which we extracted 3D human poses using MediaPipe to serve as ground-truth targets for training. These signals and their corresponding targets were then used for supervised training of an LSTM neural network. To estimate the person's position on the floor, a Kalman filter was applied to smooth the noisy SensFloor measurements. Our results demonstrate that it is possible to predict human walking poses using the proposed methods, establishing a proof-of-concept for an alternative way of activity monitoring.