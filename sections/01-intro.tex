\section{Introduction}
\label{sec:introduction}
Whenever human pose estimation is applied in various domains, including healthcare, rehabilitation and gait \& posture analysis, camera-based systems are an often used approach due to their ease of setup and efficiency \cite{avogaro2023markerlesshumanpose}. 
However, the applicability of such approaches is dependent on camera placement and the view into the room, making it susceptible to occlusion. Furthermore, the constant monitoring of people by video can be problematic regarding privacy protection. Especially in scenarios like elderly care where monitoring of residents could help caretakers to do their job. Therefore, alternative sensing approaches for monitoring human movement are required.

Considering these issues, one promising alternative for estimating human poses is floor-based sensing technology. Concretely, there are two ways to measure activity using floors: pressure- and capacitance-based. In this study, we examine SensFloor, developed by Future-Shape GmbH, as an example of a ca\-pa\-ci\-tance-based floor sensing system. As it is easy to install, SensFloor can be used in various domains, like healthcare, safety monitoring, and retail analytics. It is composed of multiple rectangular sensing areas, hereafter referred to as patches. Each patch consists of eight triangular sensing regions, referred to as fields. Each field registers changes in capacitance and outputs a signal between 127 and 255 sampled at 10~Hz, depending on the magnitude of the detected change. Since the human body contains a large amount of water, walking over the floor alters the capacitance, which SensFloor uses to detect human movement\cite[71]{lauterbach2013largeareasensorsystema}.

SensFloor has been used in several prior studies. It has been leveraged to estimate age from walking behavior \cite{hoffmann2018estimatingpersonsage}, identify individuals based on characteristic movement patterns \cite{sousa2013humantrackingidentification}, and recognize gait patterns using recurrent neural networks \cite{hoffmann2021detectingwalkingchallenges}. While these studies demonstrate the capability of SensFloor for tracking and behavioral analysis, they primarily address 2D localization or classification tasks rather than full-body pose reconstruction.
For the task of human pose estimation, different studies examined pressure-based sensors in floor covering or shoe insoles using deep-learning techniques \cite{luo2021intelligentcarpetinferring, watanabe2025p2pinsolehumanpose}. However, the feasibility of estimating full-body 3D walking poses from capacitive signals remains largely unexplored.

This study aims to address of whether 3D human walking poses can be estimated solely from SensFloor signals and to evaluate the estimation accuracy. For this purpose, we created a labeled dataset, trained a neural network that estimates poses, extracted the subjects positions from the floor, smoothed them using a Kalman filter and implemented a frontend that visualizes them.
