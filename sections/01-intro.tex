\section{Introduction}
\label{sec:introduction}


Due to population ageing and the growing shortage of care workers, AI technologies to support caregiving are increasingly required, particularly for monitoring human movement and posture to ensure patient safety. Although camera-based pose estimation has been widely studied, it is susceptible to occlusions and camera placement constraints. In addition, it raises privacy concerns. Therefore, more practical alternative approaches for monitoring human activity are necessary for deployment in real care facilities.
\todo{compress} % compress the paragraphs until here in like 2-3 sentences
\todo{compress} % Not foucssing on care giving given but we can mention it as one of the application. 

Considering these concerns, one promising alternative for automating tasks in the real care-giving is floor-based sensing technology. In this study, we focus on SensFloor as an example of a floor-based sensing system.
SensFloor is primarily used in elderly care, as it provides a less intrusive form of monitoring compared to camera-based sensors. It is a capacitive sensing floor composed of multiple rectangular sensing areas, hereafter referred to as patches. Each patch consists of eight triangular sensing regions, referred to as fields. Each field registers changes in capacitance and outputs a signal between 127 and 255 sampled at 10~Hz, depending on the magnitude of the detected change. Since the human body contains a large amount of water, walking over the floor alters the capacitance, which SensFloor uses to detect human movement [citation, A Large-Area Sensor System].

SensFloor has been used in several prior studies. Existing work has leveraged this technology to estimate age from walking behavior [4], identify individuals based on characteristic movement patterns [5], and recognize gait patterns using recurrent neural networks [3]. While these studies demonstrate the capability of SensFloor for tracking and behavioral analysis, they primarily address 2D localization or classification tasks rather than full-body pose reconstruction.
Other floor-based sensing approaches, particularly foot pressure-based systems, have also been explored for human movement analysis using neural network models, including intelligent carpets [1] and pressure-sensing insoles [2]. However, the feasibility of estimating full-body 3D walking poses specifically from SensFloor signals remains largely unexplored.


This study aims to address the research question of whether 3D human walking poses can be estimated from SensFloor signals and to evaluate the estimation accuracy.For this purpose, we created a labeled dataset, trained a neural network that estimates poses, tuned a Kalman filter to smooth the extracted positions of the estimated poses on the floor, and implemented a frontend that visualizes them. 
\todo{maybe add goal to visualize live poses in a SensFloor simulation}

%For this purpose we created a labeled dataset of people walking on a SensFloor and trained a model on that, which is the core part of a pipeline to visualize people walking on a SensFloor. 


% We construct a supervised dataset by synchronizing SensFloor signals with video data,
% from which 3D human poses are extracted using MediaPipe and used as ground-truth targets. With this dataset, we train a supervised LSTM network to estimate 3D human poses
% solely from SensFloor signals. To evaluate the performance of our model, the predicted poses are compared with the ground-truth poses, and compute the error of sum of joints . %ground-truth
% 
% The results show that the accuracy of predicted poses achieved  ~\%. Furthermore, a Kalman Filter was applied to stabilize the noisy SensFloor measurement and estimate the person’s position on the floor smoothly. 
% 
%  This study demonstrates the feasibility of estimating human poses using SensFloor without relying on cameras  and provides the proof of concept for the practical human activity monitoring using SensFloor. 



% Why are we doing this?

% Introduce sensfloor
% - What type of activities does it detect?
% - What are fields and patches

% Prior research

% Research questions

% Structure/Overview of the paper